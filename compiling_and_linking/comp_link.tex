\input{preamble}
%! suppress = LineBreak
\begin{document}
    \begin{titlepage}

        \begin{center}

            \large Санкт-Петербургский политехнический университет Петра Великого\\
            \large Институт компьютерных наук и технологий \\
            \large Кафедра компьютерных систем и программных технологий\\[6cm]

            \huge Отчет по лабораторной работе №4\\[0.5cm]
            \large по дисциплине <<Низкоуровневое программирование>>\\[0.1cm]
            \large\textbf{Раздельная компиляция}\\[5cm]

        \end{center}


        \begin{flushright}
            \begin{minipage}{0.25\textwidth}
                \begin{flushleft}

                    \large\textbf{Работу выполнил:}\\
                    \large Ильин В.П.\\
                    \large {Группа:} 35300901/10005\\

                    \large \textbf{Преподаватель:}\\
                    \large Коренев Д.А.

                \end{flushleft}
            \end{minipage}
        \end{flushright}

        \vfill

        \begin{center}
            \large Санкт-Петербург\\
            \large \the\year
        \end{center}

    \end{titlepage}
    \vfill
    \newpage
    \renewcommand\contentsname{\centerline{Содержание}}
    \tableofcontents
    \newpage
    \pagestyle{fancy}
    \setlength{\headheight}{14.5pt}
    \renewcommand{\sectionmark}[1]{\markright{#1}}
    \fancyhead[LO,RE]{Лабораторная работа №4}

    \section{Техническое задание}\label{sec:task}
    На языке C разработать функцию, реализующую задачу поиска $k$-й порядковой статистики массива.
    Поместить определение функции в отдельный исходный файл, оформить заголовочный файл.
    Разработать тестовую программу на языке C.

    Собрать программу ``по шагам''.
    Проанализировать выход препроцессора и компилятора.
    Проанализировать состав и содержимое секций, таблицы символов, таблицы перемещений и отладочную информацию, содержащуюся в объектных файлах и исполняемом файле.

    Выделить разработанную функцию в статическую библиотеку.
    Разработать make-файлы для сборки библиотеки и использующей ее тестовой программы.
    Проанализировать ход сборки библиотеки и программы, созданные файлы зависимостей.

    \section{Разработанные исходные тексты}\label{sec:listings}
    \lstinputlisting[style=cpp, label={lst:find.c}, caption=Определение функции]{listings/find.c}
    \lstinputlisting[style=cpp, label={lst:main.c}, caption=Тестовая программа]{listings/main.c}
    \lstinputlisting[style=cpp, label={lst:find.h}, caption=Заголовочный файл]{listings/find.h}
    % \lstinputlisting[style=def, label={lst:makefile}, caption=Make-файл]{listings/Makefile}

    \section{Сборка программы по шагам}\label{sec:stepbystep}
    \subsection{Препроцессирование}\label{subsec:preprocessing}
    Первым шагом является препроцессирование файлов исходного кода \verb|find.c| и \verb|main.c| в файлы \verb|find.i| и \verb|main.i| соответственно. Для этого выполним следующие команды:
    \begin{Verbatim}[breaklines=true]
riscv64-unknown-elf-gcc -march=rv32i -mabi=ilp32 -O1 -E find.c -o find.i
riscv64-unknown-elf-gcc -march=rv32i -mabi=ilp32 -O1 -E main.c -o main.i
    \end{Verbatim}
    Драйвер компилятора запускается с командами \verb|-march=rv32i -mabi=ilp32|, указывающими, что целевым является процессор с базовой архитектурой системы команд \verb|RV32I|.
    Опция \verb|-O1| указывает выполнять базовые оптимизации, а \verb|-E| — остановить процесс сборки после препроцессирования.
    Ниже представлены фрагменты полученных файлов. В полной версии файла \verb|main.i| перечислены все определения из библиотеки \verb|<stdio.h>|.
    \newpage

    \lstinputlisting[style=def, label={lst:find.i}, caption=find.i]{listings/find.i}
    \lstinputlisting[style=def, label={lst:main.i}, caption=main.i (фрагмент)]{listings/main.i}

    \subsection{Компиляция}\label{subsec:compilation}

    Далее необходимо выполнить компиляцию получившихся препроцессированных файлов, сохранив результат - сгенерированный код на языке ассемблера. Для этого воспользуемся командами:
    \begin{Verbatim}[breaklines=true]
riscv64-unknown-elf-gcc -march=rv32i -mabi=ilp32 -O1 -S -fpreprocessed find.i -o find.s
riscv64-unknown-elf-gcc -march=rv32i -mabi=ilp32 -O1 -S -fpreprocessed main.i -o main.s
    \end{Verbatim}
    Для останова процесса сборки после компиляции используется опция \verb|-S|. Флаг \verb|-fpreprocessed| (для \textit{драйвера} компилятора его указывать необязательно) говорит драйверу компилятора о том, что переданный файл уже был препроцессирован, и повторять те же действия не стоит.

    \lstinputlisting[style=riscv, label={lst:find.s}, caption=find.s]{listings/find.s}
    \vspace{2cm}
    \lstinputlisting[style=riscv, label={lst:main.s}, caption=main.s]{listings/main.s}

    Прокомментируем содержимое полученных файлов. В \verb|find.s| определяется подпрограмма \verb|find_order_statistic|, выполняющая функциональность задачи.
    В файле \verb|main.s| определяются две подпрограммы: \verb|test| и \verb|main|. Вместе они обеспечивают тестирование функциональной подпрограммы.
    В \verb|main| создаются два тестовых набора данных и для каждого из них вызывается подпрограмма \verb|test|. В ней несколько раз вызывается библиотечная подпрограмма \verb|printf| для вывода текста на экран, а также тестируемая \verb|find_order_statistic|.
    \subsection{Ассемблирование}\label{subsec:assembly}
    Для ассемблирования выполним следующие команды:
    \begin{Verbatim}[breaklines=true]
riscv64-unknown-elf-gcc -march=rv32i -mabi=ilp32 -c find.s -o find.o
riscv64-unknown-elf-gcc -march=rv32i -mabi=ilp32 -c main.s -o main.o
    \end{Verbatim}
    Опция \verb|-c| останавливает процесс сборки после ассемблирования. Стоит отметить, что в этих командах опция \verb|-O1| уже не используется, так как ассемблер (обычно) не выполняет оптимизацию.

    \subsubsection{Анализ результата}\label{subsubsec:analysis}
    Ненадолго прервем сборку и изучим полученные результаты. Для начала, рассмотрим содержимое таблиц символов полученных объектных файлов. Для этого воспользуемся другой утилитой:
    \begin{Verbatim}[breaklines=true]
riscv64-unknown-elf-objdump -t find.o main.o

SYMBOL TABLE:
00000000 l    df *ABS*  00000000 find.c
00000000 l    d  .text  00000000 .text
00000000 l    d  .data  00000000 .data
00000000 l    d  .bss   00000000 .bss
0000004c l       .text  00000000 .L2
00000040 l       .text  00000000 .L3
00000038 l       .text  00000000 .L7
00000018 l       .text  00000000 .L4
00000020 l       .text  00000000 .L5
00000000 l    d  .comment       00000000 .comment
00000000 l    d  .riscv.attributes      00000000 .riscv.attributes
00000000 g     F .text  0000005c find_order_statistic



main.o:     file format elf32-littleriscv

SYMBOL TABLE:
00000000 l    df *ABS*  00000000 main.c
00000000 l    d  .text  00000000 .text
00000000 l    d  .data  00000000 .data
00000000 l    d  .bss   00000000 .bss
00000000 l    d  .rodata        00000000 .rodata
00000000 l       .rodata        00000000 .LANCHOR0
00000000 l    d  .rodata.str1.4 00000000 .rodata.str1.4
00000000 l       .rodata.str1.4 00000000 .LC2
0000001c l       .rodata.str1.4 00000000 .LC3
00000020 l       .rodata.str1.4 00000000 .LC4
0000006c l       .text  00000000 .L2
00000054 l       .text  00000000 .L3
00000000 l    d  .comment       00000000 .comment
00000000 l    d  .riscv.attributes      00000000 .riscv.attributes
00000000 g     F .text  000000b8 test
00000000         *UND*  00000000 printf
00000000         *UND*  00000000 find_order_statistic
000000b8 g     F .text  00000098 main
    \end{Verbatim}
    Из таблицы символов видно, что символы \verb|printf| и \verb|find_order_statistic| имеют тип \verb|*UND*|. Это означает, что они использовались в ассемблерном коде, из которого был получен данный объектный файл, но не были определены; ассемблер сделал вывод о том, что символы должны быть определены где-то еще, и таким образом это отразил.

    Теперь проанализируем содержимое секций. Используем ту же утилиту с опциями \verb|-d|, позволяющей дизассемблировать содержимое файла и \verb|-M no-aliases|, требующей использовать в выводе только инструкции системы команд (но не псевдоинструкции):
    \begin{Verbatim}[breaklines=true]
riscv64-unknown-elf-objdump -d -M no-aliases -j .text find.o main.o

find.o:     file format elf32-littleriscv


Disassembly of section .text:

00000000 <find_order_statistic>:
   0:   04058663                beq     a1,zero,4c <.L2>
   4:   00259813                slli    a6,a1,0x2
   8:   01050833                add     a6,a0,a6
   c:   00000893                addi    a7,zero,0
  10:   00100313                addi    t1,zero,1
  14:   02c0006f                jal     zero,40 <.L3>

00000018 <.L4>:
  18:   00478793                addi    a5,a5,4
  1c:   01078e63                beq     a5,a6,38 <.L7>

00000020 <.L5>:
  20:   0007a683                lw      a3,0(a5)
  24:   ffc7a703                lw      a4,-4(a5)
  28:   fee6f8e3                bgeu    a3,a4,18 <.L4>
  2c:   fed7ae23                sw      a3,-4(a5)
  30:   00e7a023                sw      a4,0(a5)
  34:   fe5ff06f                jal     zero,18 <.L4>

00000038 <.L7>:
  38:   00188893                addi    a7,a7,1
  3c:   01158863                beq     a1,a7,4c <.L2>

00000040 <.L3>:
  40:   00450793                addi    a5,a0,4
  44:   fcb36ee3                bltu    t1,a1,20 <.L5>
  48:   ff1ff06f                jal     zero,38 <.L7>

0000004c <.L2>:
  4c:   00261613                slli    a2,a2,0x2
  50:   00c50533                add     a0,a0,a2
  54:   ffc52503                lw      a0,-4(a0)
  58:   00008067                jalr    zero,0(ra)

main.o:     file format elf32-littleriscv


Disassembly of section .text:

00000000 <test>:
   0:   fe010113                addi    sp,sp,-32
   4:   00112e23                sw      ra,28(sp)
   8:   00812c23                sw      s0,24(sp)
   c:   00912a23                sw      s1,20(sp)
  10:   01212823                sw      s2,16(sp)
  14:   01312623                sw      s3,12(sp)
  18:   01412423                sw      s4,8(sp)
  1c:   01512223                sw      s5,4(sp)
  20:   00050a13                addi    s4,a0,0
  24:   00058993                addi    s3,a1,0
  28:   00060a93                addi    s5,a2,0
  2c:   00060593                addi    a1,a2,0
  30:   00000537                lui     a0,0x0
  34:   00050513                addi    a0,a0,0 # 0 <test>
  38:   00000097                auipc   ra,0x0
  3c:   000080e7                jalr    ra,0(ra) # 38 <test+0x38>
  40:   02098663                beq     s3,zero,6c <.L2>
  44:   000a0413                addi    s0,s4,0
  48:   00299493                slli    s1,s3,0x2
  4c:   014484b3                add     s1,s1,s4
  50:   00000937                lui     s2,0x0

00000054 <.L3>:
  54:   00042583                lw      a1,0(s0)
  58:   00090513                addi    a0,s2,0 # 0 <test>
  5c:   00000097                auipc   ra,0x0
  60:   000080e7                jalr    ra,0(ra) # 5c <.L3+0x8>
  64:   00440413                addi    s0,s0,4
  68:   fe9416e3                bne     s0,s1,54 <.L3>

0000006c <.L2>:
  6c:   000a8613                addi    a2,s5,0
  70:   00098593                addi    a1,s3,0
  74:   000a0513                addi    a0,s4,0
  78:   00000097                auipc   ra,0x0
  7c:   000080e7                jalr    ra,0(ra) # 78 <.L2+0xc>
  80:   00050593                addi    a1,a0,0
  84:   00000537                lui     a0,0x0
  88:   00050513                addi    a0,a0,0 # 0 <test>
  8c:   00000097                auipc   ra,0x0
  90:   000080e7                jalr    ra,0(ra) # 8c <.L2+0x20>
  94:   01c12083                lw      ra,28(sp)
  98:   01812403                lw      s0,24(sp)
  9c:   01412483                lw      s1,20(sp)
  a0:   01012903                lw      s2,16(sp)
  a4:   00c12983                lw      s3,12(sp)
  a8:   00812a03                lw      s4,8(sp)
  ac:   00412a83                lw      s5,4(sp)
  b0:   02010113                addi    sp,sp,32
  b4:   00008067                jalr    zero,0(ra)

000000b8 <main>:
  b8:   fc010113                addi    sp,sp,-64
  bc:   02112e23                sw      ra,60(sp)
  c0:   000007b7                lui     a5,0x0
  c4:   00078793                addi    a5,a5,0 # 0 <test>
  c8:   0007a803                lw      a6,0(a5)
  cc:   0047a503                lw      a0,4(a5)
  d0:   0087a583                lw      a1,8(a5)
  d4:   00c7a603                lw      a2,12(a5)
  d8:   0107a683                lw      a3,16(a5)
  dc:   0147a703                lw      a4,20(a5)
  e0:   01012c23                sw      a6,24(sp)
  e4:   00a12e23                sw      a0,28(sp)
  e8:   02b12023                sw      a1,32(sp)
  ec:   02c12223                sw      a2,36(sp)
  f0:   02d12423                sw      a3,40(sp)
  f4:   02e12623                sw      a4,44(sp)
  f8:   0187a603                lw      a2,24(a5)
  fc:   01c7a683                lw      a3,28(a5)
 100:   0207a703                lw      a4,32(a5)
 104:   0247a783                lw      a5,36(a5)
 108:   00c12423                sw      a2,8(sp)
 10c:   00d12623                sw      a3,12(sp)
 110:   00e12823                sw      a4,16(sp)
 114:   00f12a23                sw      a5,20(sp)
 118:   00300613                addi    a2,zero,3
 11c:   00600593                addi    a1,zero,6
 120:   01810513                addi    a0,sp,24
 124:   00000097                auipc   ra,0x0
 128:   000080e7                jalr    ra,0(ra) # 124 <main+0x6c>
 12c:   00100613                addi    a2,zero,1
 130:   00400593                addi    a1,zero,4
 134:   00810513                addi    a0,sp,8
 138:   00000097                auipc   ra,0x0
 13c:   000080e7                jalr    ra,0(ra) # 138 <main+0x80>
 140:   00000513                addi    a0,zero,0
 144:   03c12083                lw      ra,60(sp)
 148:   04010113                addi    sp,sp,64
 14c:   00008067                jalr    zero,0(ra)
    \end{Verbatim}
    \vspace{2cm}
    Отсюда видно, что вызовы подпрограмм, например, первый вызов \verb|printf| внутри \verb|test|, из псевдоинструкции \verb|call printf| транслируется в:
    \begin{Verbatim}[breaklines=true]
 38:   00000097                auipc   ra,0x0
 3c:   000080e7                jalr    ra,0(ra) # 38 <test+0x38>
    \end{Verbatim}
    Ассемблер не имел возможности корректно определить целевой адрес перехода, поэтому сформировал некорректные нулевые значения и передал задачу исправления этих инструкций компоновщику.

    Далее рассмотрим таблицу перемещений. Она содержит информацию обо всех ``неоконченных'' инструкциях, которая передается компоновщику.
    Выполним команду с опцией \verb|-r|, выводящую таблицу перемещений:
    \begin{Verbatim}[breaklines=true]
riscv64-unknown-elf-objdump -r find.o main.o

RELOCATION RECORDS FOR [.text]:
OFFSET   TYPE              VALUE
00000000 R_RISCV_BRANCH    .L2
00000014 R_RISCV_JAL       .L3
0000001c R_RISCV_BRANCH    .L7
00000028 R_RISCV_BRANCH    .L4
00000034 R_RISCV_JAL       .L4
0000003c R_RISCV_BRANCH    .L2
00000044 R_RISCV_BRANCH    .L5
00000048 R_RISCV_JAL       .L7



main.o:     file format elf32-littleriscv

RELOCATION RECORDS FOR [.text]:
OFFSET   TYPE              VALUE
00000030 R_RISCV_HI20      .LC2
00000030 R_RISCV_RELAX     *ABS*
00000034 R_RISCV_LO12_I    .LC2
00000034 R_RISCV_RELAX     *ABS*
00000038 R_RISCV_CALL      printf
00000038 R_RISCV_RELAX     *ABS*
00000050 R_RISCV_HI20      .LC3
00000050 R_RISCV_RELAX     *ABS*
00000058 R_RISCV_LO12_I    .LC3
00000058 R_RISCV_RELAX     *ABS*
0000005c R_RISCV_CALL      printf
0000005c R_RISCV_RELAX     *ABS*
00000078 R_RISCV_CALL      find_order_statistic
00000078 R_RISCV_RELAX     *ABS*
00000084 R_RISCV_HI20      .LC4
00000084 R_RISCV_RELAX     *ABS*
00000088 R_RISCV_LO12_I    .LC4
00000088 R_RISCV_RELAX     *ABS*
0000008c R_RISCV_CALL      printf
0000008c R_RISCV_RELAX     *ABS*
000000c0 R_RISCV_HI20      .LANCHOR0
000000c0 R_RISCV_RELAX     *ABS*
000000c4 R_RISCV_LO12_I    .LANCHOR0
000000c4 R_RISCV_RELAX     *ABS*
00000124 R_RISCV_CALL      test
00000124 R_RISCV_RELAX     *ABS*
00000138 R_RISCV_CALL      test
00000138 R_RISCV_RELAX     *ABS*
00000040 R_RISCV_BRANCH    .L2
00000068 R_RISCV_BRANCH    .L3
    \end{Verbatim}
    Записи типа \verb|R_RISCV_RELAX| необходимы для оптимизации, а все остальные содержат информацию для выполнения необходимых замен.
    Дело в том, что, если точки вызова подпрограммы и сама подпрограмма находятся друг от друга достаточно близко, то вместо использования пары инструкций \verb|auipc+jalr|, позволяющей выполнять переходы в любую точку программы, можно ограничиться использованием инструкции \verb|jal|, операнд которой занимает всего 20 бит.
    \subsection{Компоновка}\label{subsec:linking}
    Для компоновки используем следующую команду:
    \begin{Verbatim}[breaklines=true]
riscv64-unknown-elf-gcc -march=rv32i -mabi=ilp32 find.o main.o -o main.out
    \end{Verbatim}
    В ней нет опции \verb|-W1,--no-relax|, а значит оптимизация, рассмотренная ранее, будет выполнена. Заметим, что компонуются файлы \textit{совместно}, а не отдельно, как это было на предыдущих этапах.

    Посмотрим на результат компоновки, выполнив следующую команду:
    \begin{Verbatim}[breaklines=true]
riscv64-unknown-elf-objdump -j .text -d -M no-aliases main.out >main.ds
    \end{Verbatim}
    В файле \verb|main.ds| интересны только некоторые фрагменты, описывающие разработанные функции:
    \begin{Verbatim}[breaklines=true]
00010144 <find_order_statistic>:
   10144:	04058663          	beq	a1,zero,10190 <find_order_statistic+0x4c>
   10148:	00259813          	slli	a6,a1,0x2
   1014c:	01050833          	add	a6,a0,a6
   10150:	00000893          	addi	a7,zero,0
   10154:	00100313          	addi	t1,zero,1
   10158:	02c0006f          	jal	zero,10184 <find_order_statistic+0x40>
   1015c:	00478793          	addi	a5,a5,4
   10160:	01078e63          	beq	a5,a6,1017c <find_order_statistic+0x38>
   10164:	0007a683          	lw	a3,0(a5)
   10168:	ffc7a703          	lw	a4,-4(a5)
   1016c:	fee6f8e3          	bgeu	a3,a4,1015c <find_order_statistic+0x18>
   10170:	fed7ae23          	sw	a3,-4(a5)
   10174:	00e7a023          	sw	a4,0(a5)
   10178:	fe5ff06f          	jal	zero,1015c <find_order_statistic+0x18>
   1017c:	00188893          	addi	a7,a7,1
   10180:	01158863          	beq	a1,a7,10190 <find_order_statistic+0x4c>
   10184:	00450793          	addi	a5,a0,4
   10188:	fcb36ee3          	bltu	t1,a1,10164 <find_order_statistic+0x20>
   1018c:	ff1ff06f          	jal	zero,1017c <find_order_statistic+0x38>
   10190:	00261613          	slli	a2,a2,0x2
   10194:	00c50533          	add	a0,a0,a2
   10198:	ffc52503          	lw	a0,-4(a0)
   1019c:	00008067          	jalr	zero,0(ra)

000101a0 <test>:
   101a0:	fe010113          	addi	sp,sp,-32
   101a4:	00112e23          	sw	ra,28(sp)
   101a8:	00812c23          	sw	s0,24(sp)
   101ac:	00912a23          	sw	s1,20(sp)
   101b0:	01212823          	sw	s2,16(sp)
   101b4:	01312623          	sw	s3,12(sp)
   101b8:	01412423          	sw	s4,8(sp)
   101bc:	01512223          	sw	s5,4(sp)
   101c0:	00050a13          	addi	s4,a0,0
   101c4:	00058993          	addi	s3,a1,0
   101c8:	00060a93          	addi	s5,a2,0
   101cc:	00060593          	addi	a1,a2,0
   101d0:	00025537          	lui	a0,0x25
   101d4:	a2050513          	addi	a0,a0,-1504 # 24a20 <__clzsi2+0x78>
   101d8:	35c000ef          	jal	ra,10534 <printf>
   101dc:	02098463          	beq	s3,zero,10204 <test+0x64>
   101e0:	000a0413          	addi	s0,s4,0
   101e4:	00299493          	slli	s1,s3,0x2
   101e8:	014484b3          	add	s1,s1,s4
   101ec:	00025937          	lui	s2,0x25
   101f0:	00042583          	lw	a1,0(s0)
   101f4:	a3c90513          	addi	a0,s2,-1476 # 24a3c <__clzsi2+0x94>
   101f8:	33c000ef          	jal	ra,10534 <printf>
   101fc:	00440413          	addi	s0,s0,4
   10200:	fe9418e3          	bne	s0,s1,101f0 <test+0x50>
   10204:	000a8613          	addi	a2,s5,0
   10208:	00098593          	addi	a1,s3,0
   1020c:	000a0513          	addi	a0,s4,0
   10210:	f35ff0ef          	jal	ra,10144 <find_order_statistic>
   10214:	00050593          	addi	a1,a0,0
   10218:	00025537          	lui	a0,0x25
   1021c:	a4050513          	addi	a0,a0,-1472 # 24a40 <__clzsi2+0x98>
   10220:	314000ef          	jal	ra,10534 <printf>
   10224:	01c12083          	lw	ra,28(sp)
   10228:	01812403          	lw	s0,24(sp)
   1022c:	01412483          	lw	s1,20(sp)
   10230:	01012903          	lw	s2,16(sp)
   10234:	00c12983          	lw	s3,12(sp)
   10238:	00812a03          	lw	s4,8(sp)
   1023c:	00412a83          	lw	s5,4(sp)
   10240:	02010113          	addi	sp,sp,32
   10244:	00008067          	jalr	zero,0(ra)

00010248 <main>:
   10248:	fc010113          	addi	sp,sp,-64
   1024c:	02112e23          	sw	ra,60(sp)
   10250:	000257b7          	lui	a5,0x25
   10254:	9f878793          	addi	a5,a5,-1544 # 249f8 <__clzsi2+0x50>
   10258:	0007a803          	lw	a6,0(a5)
   1025c:	0047a503          	lw	a0,4(a5)
   10260:	0087a583          	lw	a1,8(a5)
   10264:	00c7a603          	lw	a2,12(a5)
   10268:	0107a683          	lw	a3,16(a5)
   1026c:	0147a703          	lw	a4,20(a5)
   10270:	01012c23          	sw	a6,24(sp)
   10274:	00a12e23          	sw	a0,28(sp)
   10278:	02b12023          	sw	a1,32(sp)
   1027c:	02c12223          	sw	a2,36(sp)
   10280:	02d12423          	sw	a3,40(sp)
   10284:	02e12623          	sw	a4,44(sp)
   10288:	0187a603          	lw	a2,24(a5)
   1028c:	01c7a683          	lw	a3,28(a5)
   10290:	0207a703          	lw	a4,32(a5)
   10294:	0247a783          	lw	a5,36(a5)
   10298:	00c12423          	sw	a2,8(sp)
   1029c:	00d12623          	sw	a3,12(sp)
   102a0:	00e12823          	sw	a4,16(sp)
   102a4:	00f12a23          	sw	a5,20(sp)
   102a8:	00300613          	addi	a2,zero,3
   102ac:	00600593          	addi	a1,zero,6
   102b0:	01810513          	addi	a0,sp,24
   102b4:	eedff0ef          	jal	ra,101a0 <test>
   102b8:	00100613          	addi	a2,zero,1
   102bc:	00400593          	addi	a1,zero,4
   102c0:	00810513          	addi	a0,sp,8
   102c4:	eddff0ef          	jal	ra,101a0 <test>
   102c8:	00000513          	addi	a0,zero,0
   102cc:	03c12083          	lw	ra,60(sp)
   102d0:	04010113          	addi	sp,sp,64
   102d4:	00008067          	jalr	zero,0(ra)
    \end{Verbatim}
    \vspace{2cm}
    Во-первых, можно заметить, что пары \verb|auipc+jalr| были успешно оптимизированы до \verb|jal|, позволив сократить число инструкций. Во-вторых, в результирующих инструкциях теперь верные адреса.

    \section{Создание статической библиотеки}\label{sec:lib}
    В некоторых случаях, при большом количестве объектных файлов, удобно использовать так называемые статические библиотеки. По сути, это архивы объектных файлов, из которых компоновщик выбирает нужные.
    В случае данной задачи это не столь необходимо, поскольку используется только 2 объектных файла, а в библиотеку и то имеет смысл поместить лишь один.
    Несмотря на это, создадим статическую библиотеку.

    Для этого, используем соответствующую утилиту, которой передадим сформированный ранее объектный файл:
    \begin{Verbatim}[breaklines=true]
riscv64-unknown-elf-ar -rsc libfind.a find.o
    \end{Verbatim}
    Результирующим файлом является \verb|libfind.a|, рассмотрим его содержимое:
    \begin{Verbatim}[breaklines=true]
R:\study\lowlevelprog\c>riscv64-unknown-elf-ar -t libfind.a
find.o
    \end{Verbatim}
    Видим, что в ней действительно находится переданный файл. Теперь используем полученную библиотеку для сборки:
    \begin{Verbatim}[breaklines=true]
riscv64-unknown-elf-gcc -march=rv32i -mabi=ilp32 -O1 --save-temps main.c libfind.a -o mainwlib.out
    \end{Verbatim}
    Таблица символов полученного файла будет содержать запись о разработанной функции, а также множество символов из подключенной библиотеки \verb|<stdio.h>|:
    \begin{Verbatim}[breaklines=true]
    ...
0001027c g     F .text	0000005c find_order_statistic
    ...
    \end{Verbatim}
    \section{Разработка Make-файла}\label{sec:makedev}
    Для упрощения процесса сборки проекта удобно использовать так называемые make-файлы. По сути, они содержат консольные команды, необходимые для сборки проекта.
    Ниже представлен разработанный для данного проекта make-file, в котором создано две зависимости: для сборки проекта и удаления всех файлов, кроме исходных.
    \lstinputlisting[style=def, label={lst:makefile}, caption=Make-файл]{listings/Makefile}
    Для его использования достаточно вызвать утилиту \verb|mingw32-make| и передать ей имя make-файла.
    \begin{figure}[H]
        \centering
        \includegraphics[width=0.7\linewidth]{images/lab4_1}
        \caption{Использование make-файла}
        \label{fig:makefileusage}
    \end{figure}

    \newpage
    Вызывая в командной строке сформированный файл \verb|output.exe|, видим ожидаемый вывод:
    \begin{Verbatim}[breaklines=true]
R:\study\lowlevelprog\c>output.exe
3-th order statistic of { 4 3 5 7 2 8 } is 4
1-th order statistic of { 11 7 -4 0 } is 0
    \end{Verbatim}
    \section{Вывод}\label{sec:end}
    В ходе выполнения лабораторной работы, была реализована программа поиска $k$-й порядковой статистики массива.
    Программа была разделена на 3 файла: реализацию функциональной подпрограммы, тестовую подпрограмму, вызывающую функицональную, а также заголовочный файл.
    Проект был собран `по шагам'', во время сборки были проанализированы промежуточные файлы.
    Были разработаны и протестированы статическая библиотека, а также make-файл, упрощающие сборку программ.
\end{document}