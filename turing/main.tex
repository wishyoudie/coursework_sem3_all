\documentclass[a4paper,12pt]{article}
\usepackage[utf8x]{inputenc}
\usepackage[T1,T2A]{fontenc}
\usepackage[russian]{babel}
\usepackage{titlesec}
\titlelabel{\thetitle. }

\usepackage[dotinlabels]{titletoc}
\usepackage[left=2cm,right=2cm, top=2cm,bottom=2cm,bindingoffset=0cm]{geometry}
\usepackage{amsmath}
\usepackage{enumitem}

%% Images
\usepackage{graphicx}
\graphicspath{{images/}}

%% Custom page style
\usepackage{fancyhdr}
\renewcommand{\sectionmark}[1]{\markright{\thesection\ \#1}}
\fancypagestyle{chapter}{
    \setlength{\headheight}{14.49998pt}
    \renewcommand{\headrulewidth}{0.4pt}
    \renewcommand{\footrulewidth}{0.4pt}

    \fancyhf{}
    \fancyhead[RO]{\rightmark}
    \fancyfoot[LE,RO]{\textcopyright wishyoudie}
    \fancyfoot[C]{\thepage}
}

%% Other useful commands
\renewcommand{\bf}{\textbf} % Shortcut
\renewcommand{\it}{\textit} % Shortcut
% \setlist{noitemsep} % No spacing between list items

%% Внесение titlepage в учёт счётчика страниц
\makeatletter
\renewenvironment{titlepage} {
    \thispagestyle{empty}
    }
        \makeatother


%% Final touch
\usepackage{subfiles}

%%%%%%%%%%%%%%%%%%%%%%%%%%%%%%%%%%%%%%%%%%%%%%%%%%%%%%%%%%%%%%%%%%%%%%
\begin{document}
    \input{titlepage}
    \newpage
    \renewcommand\contentsname{\centerline{Содержание}}
    \tableofcontents
    \newpage

    \section{Техническое задание}

    Написать программу преобразования двоичного кода в унарный.
    Вид исходных данных: положительное двоичное число.
    Начальное положение головки: последняя цифра числа.

    \section{Метод решения}
    Унарные числа представляются в виде последовательности единиц, длина которой равна самому числу.
    Например: 1 -- 1, 3 -- 111, 5 -- 11111.

    Для перевода из двоичного кода будем последовательно идти по всем цифрам числа, начиная с конца.
    На каждом шаге смотрим на текущую цифру: если это $0$, то удваиваем текущее значение
    степени двойки, временно записанное слева от входного числа, а если $1$, то накапливаем
    ответ справа, после чего также увеличиваем степень.
    В конце заменяем обработанный символ на служебный ($x$) и переходим к следующему.

    В результате обработки всех цифр числа на ленте останутся только последнее значение степени двойки в унарном коде,
    изначальное число со всеми цифрами, замененными на $x$ и ответ, поэтому достаточно будет дойти до самого левого символа
    значения степени двойки, и сначала стереть степень, а затем все символы $x$.

    \vspace{5mm}
    Например, рассмотрим перевод числа $10$:
    \begin{gather*}
        1x \rightarrow 1\text{ }1x \rightarrow 11\text{ }1x \rightarrow 11\text{ }xx\text{ }11 \rightarrow 11\\
    \end{gather*}

    \section{Описание состояний}
    Алфавит:
    \begin{itemize}
        \item[$0$] -- одновременно выполняет функцию двоичной цифры и временного знака во время удвоения степени (для уменьшения используемого алфавита)
        \item[$1$] -- двоичная и унарная цифра
        \item[$x$] -- служебный символ
        \item[] Внутри подпрограммы удвоения числа будем обозначать новую цифру при помощи $x$, а $0$ -- уже удвоенную.
    \end{itemize}
    Состояния:
    \begin{itemize}
        \item[$Q_1$] -- начальное состояние итерации обработки одной цифры двоичного числа.
        Если цифра $0$ -- переход к $Q_2$, $1$ -- к $Q_{10}$.
        Если цифры кончились и головка указывает на пробел, то переход к $Q_{19}$.
        \item[$Q_2$] -- состояние для перевода головки налево от исходного числа.
        \item[$Q_3$] -- находимся на крайнем символе текущего значения степени двойки.
        Если это пробел (т.е. если рассматривалася первая цифра числа), то переходим к $Q_9$, оставляя единицу.
        Если же это $1$, то начинаем удваивать значение степени.
        Переход к $Q_4$.
        \item[$Q_4$] -- если головка смотрит на единицу, значит удвоение не завершено.
        Временно заменяем единицу на $0$ и переходим к $Q_5$.
        Если же головка смотрит на $x$, то это правая цифра, добавленная в результате удвоения, а значит все исходное число уже удвоено.
        Переходим к $Q_7$.
        \item[$Q_5$] -- идем налево до тех пор, пока не встретим пробел, на место которого ставим $x$.
        Переходим к $Q_6$.
        \item[$Q_6$] -- возвращаемся на последний необработанный символ удваимового числа.
        Начинаем новую итерацию переходом к $Q_4$.
        \item[$Q_7$] -- встаем на самый левый символ итогового числа и переходим к $Q_8$.
        \item[$Q_8$] -- заменяем все $x$ и $0$ на единицы и заканчиваем подпрограмму удвоения переходом к $Q_9$.
        \item[$Q_9$] -- после удвоения степени возвращаемся к последней необработанной цифре числа.
        Для этого пропускаем все символы, кроме служебного, которым отмечена последняя обработанная цифра.
        Найдя служебный, встаем слева от него и начинаем следующую итерацию.
        \item[$Q_{10}$] -- пропускаем исходное число, переходим к $Q_{11}$.
        \item[$Q_{11}$] -- переходим к последней цифре значения степени двойки и начинаем накапливать ответ.
        Для этого временно помечаем ее служебным символом и переходим в состояние $Q_{12}$.
        \item[$Q_{12}$] -- пропускаем все уже перенесенные в ответ единицы и, дойдя до пробела, попадаем на исходное число.
        Переходим в состояние $Q_{13}$.
        \item[$Q_{13}$] -- пропускаем число, переводя головку направо.
        Дойдя до пробела, переходим в $Q_{14}$.
        \item[$Q_{14}$] -- переходим направо от текущего ответа.
        Дойдя до пробела, дописываем единицу и переходим в $Q_{15}$.
        \item[$Q_{15}$] -- возвращаемся к исходному числу и переходим в $Q_{16}$.
        \item[$Q_{16}$] -- пропускаем исходное число и переходим в $Q_{17}$.
        \item[$Q_{17}$] -- пропускаем все перенесенные единицы, обозначенные служебными символами.
        Дойдя до конца, смотрим на символ слева от них.
        Если это $1$, то начинаем следующую итерацию переноса единицы, переход в $Q_{12}$.
        Если это пробел, то значит все единицы уже перенесены, переход в $Q_{18}$.
        \item[$Q_{18}$] -- переводим все служебные символы в значении степени назад в единицы.
        Дойдя до пробела, возвращаемся на один символ налево и переходим в $Q_4$ для удвоения степени.
        \item[$Q_{19}$] -- чистим степень, переходим к $Q_{20}$.
        \item[$Q_{20}$] -- чистим исходное число.
        Встретив единицу завершаем работу программы, так как единицы остались только в ответе.
    \end{itemize}

    \newpage

    \section{Работа программы}

    \begin{center}
        \includegraphics[scale=0.8]{lab1_1}
        \includegraphics[scale=0.8]{lab1_2}
        \includegraphics[scale=0.8]{lab1_3}
        \includegraphics[scale=0.8]{lab1_4}
        \includegraphics[scale=0.8]{lab1_5}
        \includegraphics[scale=0.8]{lab1_6}
    \end{center}
\end{document}
