\input{preamble}
\begin{document}
    \begin{titlepage}

        \begin{center}

            \large Санкт-Петербургский политехнический университет Петра Великого\\
            \large Институт компьютерных наук и технологий \\
            \large Кафедра компьютерных систем и программных технологий\\[6cm]

            \huge Отчет по лабораторной работе №2\\[0.5cm]
            \large по дисциплине <<Низкоуровневое программирование>>\\[0.1cm]
            \large\textbf{Машина EDSAC}\\[5cm]

        \end{center}


        \begin{flushright}
            \begin{minipage}{0.25\textwidth}
                \begin{flushleft}

                    \large\textbf{Работу выполнил:}\\
                    \large Ильин В.П.\\
                    \large {Группа:} 35300901/10005\\

                    \large \textbf{Преподаватель:}\\
                    \large Коренев Д.А.

                \end{flushleft}
            \end{minipage}
        \end{flushright}

        \vfill

        \begin{center}
            \large Санкт-Петербург\\
            \large \the\year
        \end{center}

    \end{titlepage}
    \vfill
    \newpage
    \renewcommand\contentsname{\centerline{Содержание}}
    \tableofcontents
    \newpage
    \pagestyle{fancy}
    \setlength{\headheight}{14.5pt}
    \renewcommand{\sectionmark}[1]{\markright{#1}}
    \fancyhead[LO,RE]{Лабораторная работа №2}

    \section{Техническое задание}

    Написать программу поиска k-й порядковой статистики массива in-place для машины EDSAC.

    \section{Метод решения}

    По определению, \textit{k-й порядковой статистикой} данного массива называется k-й элемент этого массива, если бы
    он был отсортирован.
    Будем искать ее по определению: отсортируем массив синтаксически простейшей сортировкой и возьмем элемент с индексом $(k - 1)$.

    \section{Программа Orders1}

    Для сортировки в программе используются два цикла.
    Внутренний цикл проходит по массиву и сравнивает соседние элементы: если правый элемент оказывается меньше первого, то они меняются местами;
    внешний же цикл только обеспечивает достаточное количество повторений внутреннего.
    Синтаксически, в машине EDSAC работа циклов не реализована, поэтому создадим для обоих циклов счетчики количества итераций и поместим их в ячейки 1 и 2.
    Изначально, эти счетчики будут равны (длине массива $ - 2$).
    Это число содержится в ячейке с меткой $\left<len\right>$ под номером 124.

    Внутри фрагмента реализации перестановки двух соседних элементов массива используются инструкции чтения и записи по адресам массива.
    На каждой итерации эти адреса нужно будет менять, поэтому помимо прочего, перед сортировкой необходимо инициализировать эти инструкции и куда-то их сохранить.
    Для этого, к каждой инструкции будем добавлять адрес массива, содержащийся в ячейке с меткой $\left<addr\right>$ под номером 123, предварительно сдвинув его на 1 разряд влево.
    Сформированные инструкции будем записывать в ячейки 10--13, а также обновлять их под соответствующими метками: $\left<r_1\right>, \left<r_2\right>, \left<w_1\right>, \left<w_2\right>$.

    Далее идем сам алгоритм сортировки.
    Во внешнем цикле сначала проверим условие выхода из него: считаем счетчик количества оставшихся итераций и, если он меньше нуля, то выйдем из цикла, перейдя к метке $\left<outer\_for\_exit\right>$.
    В противном же случае программа продолжит работу и войдет во внутренний цикл.
    В нем она так же сначала проверит условие выхода и, если оно не выполнится, то перейдет к чтению текущих интересующих нас элементов массива, которые будут записаны в ячейки 5 и 6.
    При записи правого элемента, для уменьшения количества действий аккумулятор не обнуляется.
    Для проверки на необходимость переставить элементы рассматривается разность $a_{j+1} - a_j \geq 0$.
    Если это условие выполняется, то осуществляется переход к метке $\left<skip\_swap\right>$, иначе осуществляется перестановка.

    В конце каждой итерации внутреннего цикла обновляются инструкции чтения и записи -- к каждой прибавляется сдвинутая на один разряд влево единица, а также единица вычитается из счетчика оставшихся итераций.

    Если же во внутреннем цикле сработал переход к метке $\left<inner\_for\_exit\right>$, то значит на этом шаге внешнего цикла все необходимые итерации внутреннего уже были выполнены, и надо переходить к следующей.
    Для этого, во-первых, восстанавливается счетчик оставшихся итераций внутреннего цикла, во-вторых, единица вычитается из счетчика для внешнего цикла, и, в-третьих, при помощи сохраненных заранее в ячейки 10--13копий, восстанавливаются инструкции чтения и записи.
    Осуществляется переход в начало внешнего цикла.

    После перехода к метке $\left<outer\_for\_exit\right>$, циклы закончатся, а массив будет отсортирован.
    Все, что остается -- найти $a_{k-1}$.
    Для этого в аккумулятор добавляется адрес массива, после чего к нему прибавляется $k$ и вычитается единица.
    После этого, содержимое аккумулятора сдвигается на один разряд влево и прибавляется к инструкции чтения ответа: $\left<r_{ans}\right>$.

    Ответ записывается в ячейку $\left<res\right>$ под номером 126, и программа завершает работу.

    \subsection{Код программы Orders1}
    \begin{flushleft}
        \includegraphics[width=.7\linewidth]{images/lab2)1}
        \newpage
        \includegraphics[width=.7\linewidth]{images/lab2_2}
        \includegraphics[width=.7\linewidth]{images/lab2_3}
    \end{flushleft}
    \subsection{Работа программы Orders1}
    \begin{figure}[H]
        \centering
        \includegraphics[width=\linewidth]{images/lab2_4}
        \label{fig:res_1}
    \end{figure}

    \section{Программа Orders2}
    Алгоритм программы для загрузчика Initial Orders 2 остается таким же.
    Изменения состоят в том, что Initial Orders 2 поддерживает создание подпрограмм (поэтому этап сортировки массива вынесен в отдельную подпрограмму) и относительную адресацию, из-за чего комментарий перед каждой инструкцией теперь содержит не только абсолютный номер ячейки, но и ее относительный адрес.
    Главная же подпрограмма, из которой происходит вызов сортировки, предварительно записывает в ячейки 5 и 6 входные данные: адрес сортируемого массива и его длину.
    После этого вызывается подпрограмма сортировки, а после ее завершения в главной подпрограмме аналогично Orders1 находится $a_{k-1}$.

    Ответ находится в ячейке $\left<ans\right>$ с адресом 162.
    \subsection{Код программы Orders2}
    \begin{flushleft}
        \includegraphics[width=.7\linewidth]{images/lab2_6}
        \newpage
        \includegraphics[width=.7\linewidth]{images/lab2_7}
        \includegraphics[width=.5\linewidth]{images/lab2_8}
    \end{flushleft}
    \subsection{Работа программы Orders2}
    \begin{figure}[H]
        \centering
        \includegraphics[width=\linewidth]{images/lab2_5}
        \label{fig:res_2}
    \end{figure}
\end{document}