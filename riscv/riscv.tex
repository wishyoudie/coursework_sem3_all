\input{preamble}
\begin{document}
    \begin{titlepage}

        \begin{center}

            \large Санкт-Петербургский политехнический университет Петра Великого\\
            \large Институт компьютерных наук и технологий \\
            \large Кафедра компьютерных систем и программных технологий\\[6cm]

            \huge Отчет по лабораторной работе №3\\[0.5cm]
            \large по дисциплине <<Низкоуровневое программирование>>\\[0.1cm]
            \large\textbf{Программирование RISC-V}\\[5cm]

        \end{center}


        \begin{flushright}
            \begin{minipage}{0.25\textwidth}
                \begin{flushleft}

                    \large\textbf{Работу выполнил:}\\
                    \large Ильин В.П.\\
                    \large {Группа:} 35300901/10005\\

                    \large \textbf{Преподаватель:}\\
                    \large Коренев Д.А.

                \end{flushleft}
            \end{minipage}
        \end{flushright}

        \vfill

        \begin{center}
            \large Санкт-Петербург\\
            \large \the\year
        \end{center}

    \end{titlepage}
    \vfill
    \newpage
    \renewcommand\contentsname{\centerline{Содержание}}
    \tableofcontents
    \newpage
    \pagestyle{fancy}
    \setlength{\headheight}{14.5pt}
    \renewcommand{\sectionmark}[1]{\markright{#1}}
    \fancyhead[LO,RE]{Лабораторная работа №3}

    \section{Техническое задание}

    Написать программу поиска k-й порядковой статистики массива in-place при помощи системы команд RISC-V.

    \section{Метод решения}

    По определению, \textit{k-й порядковой статистикой} данного массива называется k-й элемент этого массива, если бы
    он был отсортирован.
    Будем искать ее по определению: отсортируем массив синтаксически простейшей сортировкой и возьмем элемент с индексом $(k - 1)$.

    \section{Программа 1}
    В первом варианте программы (файл \verb|first.s|) все вычисления происходят в одной функции.
    В режиме отладки программы вызывается при помощи команды
    \begin{verbatim}
    jupiter -debug first.s
    \end{verbatim}
    \subsection{Реализация программы 1}
    \begin{figure}[H]
        \centering
        \includegraphics[width=\linewidth]{images/lab3_1}
        \caption{Реализация программы 1}
        \label{fig:code_1}
    \end{figure}

    \subsection{Работа программы 1}
    \begin{figure}[H]
        \centering
        \includegraphics[width=\linewidth]{images/lab3_2}
        \caption{Работа программы 1}
        \label{fig:debug_1}
    \end{figure}

    \section{Программа 2}
    Во втором варианте программа реализована при помощи двух подпрограмм: в одной содержится функция \verb|main| (файл \verb|main.s|), вызывающая подпрограмму, подсчитывающую ответ (ее реализация представлена в файле \verb|find_order_statistic.s|).
    Стартовая метка, на которой происходит вызов главной подпрограммы, находится в файле \verb|second.s|.
    В режиме отладки программа вызывается при помощи команды:
    \begin{verbatim}
        jupiter -debug -start __start second.s main.s find_order_statistic.s
    \end{verbatim}
    \subsection{Реализация программы 2}
    \begin{figure}[H]
        \centering
        \includegraphics[width=0.3\linewidth]{images/lab3_3_1}
        \caption{Файл тестовой программы}
        \label{fig:code_2}
    \end{figure}

    \begin{figure}[H]
        \centering
        \includegraphics[width=\linewidth]{images/lab3_3_2}
        \caption{Файл с основной подпрограммой}
        \label{fig:code_3}
    \end{figure}

    \begin{figure}[H]
        \centering
        \includegraphics[width=\linewidth]{images/lab3_3_3}
        \caption{Файл с функциональной подпрограммой}
        \label{fig:code_4}
    \end{figure}



    \subsection{Работа программы 2}
    \begin{figure}[H]
        \centering
        \includegraphics[width=\linewidth]{images/lab3_4}
        \caption{Работа программы 2}
        \label{fig:debug_2}
    \end{figure}

\end{document}